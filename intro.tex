\chapter{Introduction}

This setting book will to evolve a vision for a Science-Fiction/Fantasy
universe that has been lingering at the back of my head for a while. Here I am
foremost focusing on worldbuilding, trying to write a manifesto that captures
the essence of this setting. In doing so, I am trying not to stray too much
into unimportant details -- the scope of this world will be huge and complex,
way larger than a single person could ever fill with life.
It might well be, that due to the scope of this setting, the characters and
places of one story or game round might never meet or even be knowable to those
of another. Rather than trying the tantalizing task of constructing a perfectly
consistent world and history full of unarguable details, I will only sketch a
rough background of world and history -- as subjective and incomplete as the
world we live in.
This background is more supposed to convey specific moods and emotions, motives
and ideas. In doing so, I hope to give some coherence to all stories happening
in this setting on a more abstract level.\\

To get a first grasp on the ideas and moods I am trying to convey, let me list
some of these:

\begin{itemize}
\item The universe is vast and indifferent to mankind ... even while mankind
conquers the stars.
\item Mankind itself is vastly complex in all the ways culture and the human
psyche evolved ... across time as well as space.
\item History repeats itself, and the basic conflicts, cravings and fears of man
stay the same.
\item There will be no Utopia to end mankind's struggles -- We will always face
challenges.
\item There is no real progress -- At most change and diversification.
\item Nature will always hold mystery, however far we evolve.
\item Advanced technology holds the same kind of mystery to those who do not
understand it.
\end{itemize}

An important part of Science-Fiction is extrapolating a possible future for
mankind to look at philosophies and problems that matter \emph{today}. Often
this aspect happens with regard to the so called 'Novum' a very specific societal
development or technology that defines the course of history. In a similar vein
I will define three rules that are used as condensation nucleus for the themes
listed above:

\begin{enumerate}
\item \textbf{There are no significant gamechangers beyond the laws of physics known
today.}\\
While there are infinitely complex details to study in the universe, the basics
stand -- especially concerning the way humans can interact with nature.
Lightspeed is the definitive limit of communication, conservation of energy
and momentum holds and, every action produces a reaction.
\emph{We are moving slowly, and nothing comes without a price.}
\item \textbf{The universe is generally fertile to life, but truly intelligent life
is very unlikely.}\\
As far as I can judge, this is mostly in line with what we can conclude today,
while it also conveniently avoids the problem that this universe might become
too complex. Mankind will be complex enough in and by itself, how much harder
would it be to construct a similar convincing, but still alien society?
And foremost it serves the mood of cosmic loneliness. \emph{We are alone, and
the only ones to help us will be ourselves}
\item \textbf{The culture known today is remembered across millenia, although in
sometimes strangely deformed ways.}\\
This might be the largest stretch amongst these rules, but it serves the
narrative purpose to keep the world relatable to todays world.
While it seems unlikely that people in 10.000 years care about todays gods and
customs it simplifies the task for the author to come up with a diverse range
of human behaviour.
\end{enumerate}

To give some credit as well as inspiration, these and similar ideas have been
explored in a vast number of works. The tropes of technological progress
slowing to a crawl and mankind staying fractured into poor and rich -- or low-tech
and high-tech -- are standard tropes of cyberpunk and end-time universes. Here
I am mostly influenced by the \textit{Fallout}-games, the
\textit{Warhammer 40k}-universe, and the movies \textit{Ghost in the Shell} and
\textit{Blade Runner}. Especially regarding the role of low-end technology, I
should mention the more hard-sci-fi series \textit{The Expanse} and
\textit{Firefly}.

The feeling of appreciating the mystic beauty of a vast, indecipherable universe
is best summed up in the term \emph{sense of wonder}. Nowhere has it been
captured as masterfully as in the works of Arthur C. Clarke. He also expanded
this notion to technology beyond the human capability of understanding and
coined the genre-defining phrase: \emph{Any sufficiently advanced technology is
indistinguishable from magic.}

Keeping all of this in mind I want to construct a world that reaches a space
opera scope similar to \textit{Star Wars} or \textit{Warhammer 40k} but stays
technologically down to earth by not blatantly disobeying laws of physics. One
should be able to imagine a logic route of history to the
state at its telling. Yet there is no need to flesh out every single planet and
political person. Staying scientifically grounded means that travelling to the
stars will be an endeavour of huge effort and demand incredible personal
sacrifices due to the time-scales alone. Individual stories or game sessions may
change the fate of whole societies or planetary populations, yet still be
insignificant in the big picture. Cultures will clash in apocaliptic ways, yet
in the end everything is dust in the wind...

One last note to the reader: The informations in the following chapters are
mostly to be considered the inner clockworks driving this worlds evolution and
no single character knows everything, or even a small fraction of these
happenings. These events happen beyond the timescales of most people's
historical or spatial horizon and only very few enlightened people are able to
grasp the scope of mankind's history. Therefore, especially if this setting is
used in roleplay, players should only read as much as needed and be able to
strictly separate player- from character-knowledge. If used in a story, the
big strokes of history should be revealed very scarcely and such a reveal should
mark a major plot point.
